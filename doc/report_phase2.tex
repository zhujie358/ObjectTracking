\documentclass[11pt]{article} %Basic document type
\usepackage{times} %Font 
\usepackage{amsmath} %For matrices
\usepackage{amssymb}
\usepackage[section]{placeins} %Allows FloatBarrier command
\usepackage[utf8]{inputenc} %Font encoding
\usepackage[margin=1.0in]{geometry} %Adjust margins
\usepackage{graphicx} %Allows picture import
\usepackage{pdfpages} %Including pdf files 
\usepackage{setspace}
\usepackage{fancyhdr}
\usepackage{appendix}
\usepackage{mcode}
\graphicspath{{images/}} %Allows picture impor
\pagestyle{fancy}
\fancyhf{}
\fancyhead[L]{A Real-Time Object Tracking System}
\fancyhead[R]{ECSE 457}
\fancyfoot[C]{\thepage}
\renewcommand{\headrulewidth}{1pt}
\renewcommand{\footrulewidth}{1.5pt}
\newcommand{\HRule}[1][\medskipamount]{\par
  \vspace*{\dimexpr-\parskip-\baselineskip+#1}
  \noindent\rule{\linewidth}{0.2mm}\par
  \vspace*{\dimexpr-\parskip-.5\baselineskip+#1}}
\begin{document}
\begin{titlepage}
\begin{center}
\textsc{\huge McGill University}\\[1.5cm]
\textsc{\LARGE Department of Electrical \& Computer Engineering}\\[1.5cm]
\textsc{\Large ECSE 457 - Final Report}\\[3cm]
\HRule
{\huge \bfseries Research \& Development of a Real-Time Object Tracking System \\[.3cm] }
\HRule 
\vspace{1.5cm}
\noindent
\begin{minipage}{0.4\textwidth}
\begin{flushleft} \large
\emph{\Large Authors:}\\
\vspace{.2cm}
Benjamin \textsc{Brown} \\
\textit{benjamin.brown2@mail.mcgill.ca} \\
260450182 \\
\vspace{.2cm}
Taylor \textsc{Dotsikas} \\
\textit{taylor.dotsikas@mail.mcgill.ca} \\
260457719
\end{flushleft}
\begin{flushleft} \large
\emph{\Large Supervisors:}\\
\vspace{.2cm}
Warren \textsc{Gross, Prof.}\\
\vspace{.2cm}
Arash \textsc{Ardakani} 
\end{flushleft}
\begin{flushleft} \large
\emph{\Large Sponsored By:}\\
\vspace{.2cm}
\textsc{Analog Devices}\\
\end{flushleft}
\end{minipage}%
\end{center}
\end{titlepage}
\pagebreak
\section*{Abstract}
Hello World
\section*{Acknowledgments}
Hello World
\pagebreak
\tableofcontents
\pagebreak
\section{Abbreviations \& Notation}
\begin{itemize}
\item[] FPGA - Field Programmable Gate Array
\item[] VGA - Video Graphics Array
\end{itemize}
\section{Introduction}
Hello World
\section{Background}
This section contains the prerequisite information regarding video tracking needed to understand the system architecture and design. For background information regarding the basics of video processing, Kalman filtering, fixed-point representation, and optical flow, please see \cite{15}.
\subsection{Determining Position}
The algorithm implemented in software, and presented in \cite{15}, for determining the $(x,y)$ position of an object in the delta frame used a rastor scan technique to determine the leftmost, rightmost, top, and bottom pixels. By intersecting two lines formed between these points, the center of the object can be estimated. It was found that despite the success of this algorithm in software, it would not be conducive to hardware implementation.
\subsection{Video Pipeline}
Describe a video pipeline and common components.
\subsection{The VGA Interface}
Discuss the timing signals 
\newpage
\begin{thebibliography}{11}
\bibitem{1}
F. Roth, “Using low cost FPGAs for realtime video processing”, M.S. thesis, Faculty of Informatics, Masaryk University, 2011.
\bibitem{2}
A. Saeed et al., “FPGA based Real-time Target Tracking on a Mobile Platform,” in 2010 International Conference on Computational Intelligence and Communication Networks, 2010, pp. 560-564.
\bibitem{3}
”Video and Image Processing Design Using FPGAs.” Altera. 2007. January 2014. 
http://www.altera.com/literature/wp/wp-video0306.pdf  
\bibitem{4}
E. Trucco and A. Verri, “Chapter 8 - Motion,” in Introductory Techniques for 3D Computer, pp. 177- 219.
\bibitem{5}
S.A. El-Azim et al., “An Efficient Object Tracking Technique Using Block-Matching Algorithm”, in Nineteenth National Radio Science Conference, Alexandria, 2002, pp. 427 - 433.
\bibitem{6} 
Caner et al., “An Adaptive Filtering Framework For Image Registration”, IEEE Trans. Acoustics, Speech, and Signal Processing, vol. 2, no. 2, 885-888. March, 2005. 
\bibitem{7}
Yin et al. \textit{Performance Evaluation of Object Tracking Algorithm} [Online]. Available: http://dircweb.kingston.ac.uk/ 
\bibitem{8}
G. Shrikanth, K. Subramanian, “Implementation of FPGA-Based object tracking algorithm,” Electronics and Communication Engineering Sri Venkateswara College of Engineering, 2008.
\bibitem{9}
E. Pizzini, D. Thomas, “FPGA Based Kalman Filter,” Worcester Polytechnic Institute, 2012.
\bibitem{10}
M. Shabany. (2011, December 27). \textit{Floating-point to Fixed-point Conversion} [Online]. Available: http://ee.sharif.edu/~digitalvlsi/Docs/Fixed-Point.pdf
\bibitem{11} 
N. Devillard. (1998, July). \textit{Fast median search: an ANSI C implementation} [Online]. Available: http://ndevilla.free.fr/median/median.pdf
\bibitem{12}
D. Kohanbash. (2014, January 30). \textit{Kalman Filtering - A Practical Implementation Guide (with code!)} [Online]. Available: http://robotsforroboticists.com/kalman-filtering/
\bibitem{13} Recommendation ITU-R BT.656-5 (12/2007). “Interface for digital component video signals in 525-line and 625-line television systems operating at the 4:2:2 level of Recommendation ITU-R BT.601". 
\bibitem{14} ISSI Datasheet: “1M x 16 High-Speed Asynchronous CMOS Static RAM with 3.3V Supply".
\bibitem{15} B. Brown \& T. Dotsikas. 
\end{thebibliography}
\newpage
\appendix
\appendixpage
\end{document}